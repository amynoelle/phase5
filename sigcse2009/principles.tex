\section{Design Principles}
\label{sec:designprinciples}

Pintos's projects are built on a number of principles.

\paragraph{Read before you Code}
Each project involves a significant amount of reading code before
students write the first line of code.
Because software maintenance constitutes the vast majority of all
software development efforts~\cite{Boehm1981Software}, this setup mirrors the 
environment in which most software engineers work.
Simultaneously, we limit the amount students have to read
by encapsulating lower layers, such as device drivers.
We went to great lengths to write the entire Pintos baseline code,
and in particular the portions students must read, in a style that shows,
by example, the coding style we expect from students.  
% cut for length
%This style includes purely syntactical convention such as the choice of the
%GNU indentation style, and extends to commenting style and naming conventions.
We continuously refined the internal code documentation over several
semesters, focusing on those 
portions that initially proved difficult to understand or confusing.

\paragraph{Maximize Creative Freedom}
OS design involves a tremendous amount of creative freedom, both in the
choice of algorithms and data structures.  Our projects are designed to
stimulate creativity by avoiding the prescription of specific approaches
to accomplish each project's goals.  Instead, students design their
own data structures and associated algorithms as much as possible.

\paragraph{Practice Test-driven Development}
%Test-driven development~\cite{Edwards}
Each project includes a large number of test cases that is accessible
to students, as shown in  Table~\ref{table:tests}.  
They must implement the API that is exercised by these test cases.  
Students are encouraged to add their own test cases.

\paragraph{Work in a Team}
The projects presented in this paper are designed to be accomplished by teams of 
2-4 students.  Working in a team provides an environment that more closely resembles
industrial software development, and it provides a way for students to brainstorm and
implement together.  In addition, we teach and require the use of group collaboration tools,
notably shared source code version control systems such as CVS.

\paragraph{Justify your Design}
Design justification and rationale is as important for learning as creating an artifact 
that fulfills a set of given requirements.  We designed a set of structured questionnaires 
in which students describe their design and discuss choices and trade-offs they made.

\paragraph{Provide a Reproducible, Manageable Environment}
Operating Systems are inherently concurrent environments, which can be difficult
to debug.  For educational use, we must provide an environment that is
manageable and reproducible, which we do by providing the option
of running Pintos in a simulated, fully deterministic environment.  
As a result, Pintos kernels can be debugged in a manner that
is substantially similar to how user programs are being debugged.

\paragraph{Include Analysis Tools}
Static and dynamic analysis tools are now being widely used; an OS course should
be no exception.  In Section~\ref{sec:dynamicanalysis}, we describe how we 
extended the QEMU emulator~\cite{Bellard2005QEMU} to 
perform tailored analyses that find errors such as race conditions.

\paragraph{Provide Extensive and Structured Documentation}
If using an instructional system requires too much undocumented knowledge,
the system is often not shared or falls into disuse because the learning curve
for instructors is too steep and training teaching assistants is difficult.
Pintos includes an extensive 129 page manual, a sample solution,
and grading instructions for teaching assistants.  The project documentation 
highlights sections students must read from sections that merely provide supplemental information.
%Even though Pintos uses an existing and complex architecture, our experience indicates
%that the manual is sufficient for most students.


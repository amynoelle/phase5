\section{Design Principles}
\label{sec:designprinciples}

The Pintos series of projects are built on a number of principles.

\paragraph{Read before You code.}
Each project involves a significant amount of reading code before
students write the first line of their code.  
Because software maintenance constitutes the vast majority of all
software development efforts~\cite{Boehm1981Software}, this setup mirrors the 
environment in which most software engineers work.
We went to great lengths to write the entire Pintos baseline code,
and in particular the portions students will read, in a style that shows,
by example, the coding style we expect from students.  This style
includes purely syntactical convention such as the choice of the
GNU indentation style, and extends to commenting style and naming 
conventions.  During the semesters in which Pintos was used, we
continuously refined the internal code documentation, focusing on those 
portions that initially proved difficult to understand or confusing.

\paragraph{Maximize Creative Freedom}
OS design involves a tremendous amount of creative freedom, both in the
choice of algorithm and data structures.  Our projects are designed to
stimulate creativity by avoiding the prescription of specific approaches
to accomplish each project's goals.  Instead, students must design their
own data structures and associated algorithms as much as possible.

\paragraph{Practice Test-driven Development}
%Test-driven development~\cite{Edwards}
Each project includes a large number of test cases that is accessible
to students.  
They must implement the API that is exercised by these test cases.
Students are encouraged to add their own test cases.

\paragraph{Work in a Team}
The projects presented in this paper are designed to be accomplished by teams of 
2-4 students.  Working in a team provides an environment that more closely resembles
industrial software development, and it provides a way for students brainstorm and
implement together.  In addition, we teach and require the use of group collaboration tools,
notably shared source code version control systems such as CVS.

\paragraph{Justify your Design}
Design justification and rationale is as important for learning as creating an artifact 
that fulfill a set of given requirements.  We designed a set of structured questionnaires 
in which students describe their design and discuss choices and trade-offs they made.

\paragraph{Provide a Reproducible, Manageable Environment}
Operating Systems are inherently concurrent environments, which can be difficult
to debug.  For educational use, we must provide an environment that is
manageable and reproducible, which is given by the option
of running Pintos in a simulated environment eliminates this
non-determinism.  As a result, Pintos kernels can be debugged in a manner that
is substantially similar to how user programs are being debugged.

\paragraph{Provide analysis tools.}
Static and dynamic analysis tools are now widely being used; an OS course should
be no exception.  We have extended the Qemu emulator that perform tailored
analyses that can point out errors such as race conditions.

